% !TEX TS-program = pdflatex
% !TEX encoding = UTF-8 Unicode

% This is a simple template for a LaTeX document using the "article" class.
% See "book", "report", "letter" for other types of document.

\documentclass[8pt]{extarticle} % use larger type; default would be 10pt

\usepackage[utf8]{inputenc} % set input encoding (not needed with XeLaTeX)

%%% Examples of Article customizations
% These packages are optional, depending whether you want the features they provide.
% See the LaTeX Companion or other references for full information.

%%% PAGE DIMENSIONS
\usepackage{geometry} % to change the page dimensions
\geometry{a4paper} % or letterpaper (US) or a5paper or....
\geometry{margin=0.5in} % for example, change the margins to 2 inches all round
% \geometry{landscape} % set up the page for landscape
%   read geometry.pdf for detailed page layout information

\usepackage{graphicx} % support the \includegraphics command and options

\usepackage[parfill]{parskip} % Activate to begin paragraphs with an empty line rather than an indent

%%% PACKAGES
\usepackage{booktabs} % for much better looking tables
\usepackage{array} % for better arrays (eg matrices) in maths
\usepackage{paralist} % very flexible & customisable lists (eg. enumerate/itemize, etc.)
\usepackage{verbatim} % adds environment for commenting out blocks of text & for better verbatim
\usepackage{subfig} % make it possible to include more than one captioned figure/table in a single float
\usepackage{verbatim}
\usepackage{hyperref}
% These packages are all incorporated in the memoir class to one degree or another...

%%% HEADERS & FOOTERS
\usepackage{fancyhdr} % This should be set AFTER setting up the page geometry
\pagestyle{fancy} % options: empty , plain , fancy
\renewcommand{\headrulewidth}{1.5pt} % customise the layout...
\renewcommand{\footrulewidth}{1.5pt} % customise the layout...
\lhead{Logico tne Python Logic Interpereter}\chead{About}\rhead{Phil 201}
\lfoot{}\cfoot{\thepage}\rfoot{}

%%% ToC (table of contents) APPEARANCE
\usepackage[nottoc,notlof,notlot]{tocbibind} % Put the bibliography in the ToC
\usepackage[titles,subfigure]{tocloft} % Alter the style of the Table of Contents
\renewcommand{\cftsecfont}{\rmfamily\mdseries\upshape}
\renewcommand{\cftsecpagefont}{\rmfamily\mdseries\upshape} % No bold!

%%% END Article customizations

%%% The "real" document content comes below...

\setlength{\parindent}{0cm}

\begin{document}
 
\section{What is Logico?}

Logico is a Python implementation of the robot named Logico that interperets categorical propositions      and categorical syllogisms. This is exactly what the Python implementation does.

\section{Simple Stuff and Limitations}
-
However, the Python implementation cannot describe truth. For example, it cannot tell you whether or not all cats are dogs; you must tell it that. In a sense, it is not telling you whether 3 is a square root, but telling you what 2+3 is instead. So the Logico Python implementation can tell validity, but not truth. It can tell when a syllogism is invalid.

Here is the method you would take to make a proposition in general:

\begin{verbatim}
>>> A=Propos(subject,predicate,lettertype,truth_value)
\end{verbatim}

For instance,

\begin{verbatim}
>>> A=Propos("subject","dogs","A",True)
\end{verbatim}

The quotation marks {\bf{are needed}}. And a syllogism can be completed using this command (again, in general):

\begin{verbatim}
>>> S=Syllog(A,B,C)
\end{verbatim}  

where A, B, and C are already-defined propositions.

There are many operations you can peform from here, but most of them have this general syntax.

\begin{verbatim}
A.propos_method()
\end{verbatim}
where A is an already-defined propositon. Notice the ()'s. Likewise, a syllogism method can be called like so:

\begin{verbatim}
S.syllog_method()
\end{verbatim}

where S is an already-defined syllogism.

To find the list of all of the operations, type \verb|help(Propos)| for the propositions and \verb|help(Syllog)| for the syllogisms.

\section{Help Text}

\begin{verbatim}
Help on class Propos in module __main__:

class Propos(builtins.object)
 |  Class for a categorical proposition from the Aristotelian standpoint.
 |
 |  Methods defined here:
 |
 |  __init__(self, sub, pred, lettype, tVal)
 |      __init__(str,str,str,bool) -> Propos
 |      Propositon constructor. Takes the subject, predicate, categorical proposition type,
 |	 and truth value.
 |
 |  __str__(self)
 |      Propos.__str__() -> str
 |      Class printer. Will put "F" in front of the proposition if it is false.
 |
 |  chgQuality(self)
 |      Propos.chgQuality() -> None
 |      Changes the quality of the proposition.
 |
 |  chgQuantity(self)
 |      Propos.chgQuantity() -> None
 |      Changes the quantifier of the proposition.
 |
 |  contradictory(self)
 |      Propos.chgQuality() -> None
 |      Performs contradiction on the propositon.
 |
 |  contrapose(self)
 |      Propos.contrapose() -> None
 |      Performs contradiction on the propositon.
 |      If proposition does not satisfy the conditions, returns "Illicit contraposition."
 |
 |  contrary(self)
 |      Propos.contrary() -> None
 |      Performs Aristotelian contrary on the propositon.
 |      If proposition does not satisfy the conditions, returns "Illicit contrary."
 |
 |  convert(self)
 |      Propos.convert() -> None
 |      Performs conversion on the propositon.
 |      If proposition does not satisfy the conditions, returns "Illicit conversion."
 |
 |  getEngName(self)
 |      Propos.getEngName() -> str
 |      Same as __str__, but ignores truth.
 |
 |  getQuality(self)
 |      Propos.getQuality() -> str
 |      Prints quality of proposition.
 |
 |  getQuantity(self)
 |      Propos.getQuantity() -> str
 |      Prints quantity of proposition.
 |
 |  getTermsDist(self)
 |      Propos.getTermsDist() -> list
 |      Lists the terms distributed by the propositon.
 |
 |  obvert(self)
 |      Propos.obvert() -> None
 |      Performs obversion on the propositon.
 |
 |  subalt(self)
 |      Propos.subalt() -> None
 |      Performs Aristotelian subalternation on the propositon.
 |      If proposition does not satisfy the conditions, returns "Illicit subalternation."
 |
 |  subcontrary(self)
 |      Propos.subcontrary() -> None
 |      Performs Aristotelian subcontrary on the propositon.
 |      If proposition does not satisfy the conditions, returns "Illicit subcontrary."
 |
 |  ----------------------------------------------------------------------
\end{verbatim}

\begin{verbatim}
Help on class Syllog in module __main__:

class Syllog(builtins.object)
 |  A class for a categorical syllogism.
 |  
 |  Methods defined here:
 |  
 |  __init__(self, majorPrem, minorPrem, conclus)
 |      __init__(Propos,Propos,Propos) -> Syllog
 |      Class constructor. Takes three propositions as input.
 |  
 |  __str__(self)
 |      Propos.__str__() -> str
 |      Class printer. Bar adjusts itself to length of the longest
 |      propositions. Also ignores putting F for false propositions.
 |  
 |  getFig(self)
 |      Propos.getFig() -> int
 |      Gives the figure of the syllogism as an Int.
 |  
 |  getMiddleTerm(self)
 |      Propos.getMiddleTerm() -> str
 |      Gives the middle term of the syllogism.
 |  
 |  getMood(self)
 |      Propos.getMood() -> str
 |      Gives the mood of the syllogism.
 |  
 |  isValid(self)
 |      Propos.getFig() -> bool
 |      Tests for validity. IT DOES NOT TEST FOR SOUNDNESS.
 |  
 |  ----------------------------------------------------------------------
\end{verbatim}


\section{Downloading Python for your Computer}

Please visit \url{https://www.python.org/downloads/}. Then click "Download Python 3.4.1."

\end{document}
